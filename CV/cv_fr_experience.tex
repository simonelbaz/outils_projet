%\setlength\LTleft{\parindent}
%\setlength\LTright{\fill}
%\setlength\LTleft{0pt}
%\setlength\LTright{0pt}
\begin{longtable}{p{0.3\linewidth}X}
\hline
\textbf{Exp\'erience professionelle} \\
\hline
Janvier 2012 -- Pr\'esent & \emph{Migration du SI des permis de conduire \`a points \`a l'Agence Nationale des Titres S\'ecuris\'es (www.ants.interieur.gouv.fr)} \\

 &
\begin{itemize}
\item Participation \`a la phase d'int\'egration de la migration du SNPC (ancien syst\`eme de gestion des permis de conduire) vers Faeton
\item R\'edaction du cahier de recette et suivi de l'avancement de la validation \`a partir de Quality Center de HP
\item Responsable de l'\'equipe de validation (3 personnes)
\end{itemize} \\

Octobre 2011 -- D\'ecembre 2011 & \emph{Revenue Management chez Air France - KLM} \\
& \begin{itemize}
\item         Traitement de l'historique des r\'eservations de billets Air France - KLM sur une p\'eriode de 3 ans : 2 milliards d'\'ev\'enements de booking de billets Air France - KLM trait\'es en Perl et ETL Talend en moins de 3h 30 minutes
\end{itemize} \\

Ao\^ut 2011 -- Septembre 2011& \emph{Migration du SI de gestion des stocks des boutiques Orange} \\
& \begin{itemize}
\item         Participation \`a la phase d'int\'egration du progiciel Torex de gestion des stocks de boutiques dans le projet Bestar
\item         Participation \`a la phase d'int\'egration de d\'eveloppements effectu\'es avec l'ETL Talend
\item         Responsable de l'\'equipe de validation (2 personnes)
\end{itemize} \\

Mai 2010 -- Juillet 2011 & \emph{Migration du SI mobile chez SFR, Grand Public} \\
& \begin{itemize}
\item         Participation \`a la phase d'int\'egration du progiciel de billing Oracle BRM dans le projet BIOS
\item         Participation \`a la phase d'int\'egration de d\'eveloppements effectu\'es avec l'ETL Datastage
\item         D\'eveloppement d'\'evolutions dans le domaine de la valorisation (pipeline BRM)
\end{itemize} \\

Ao\^ut 2009 -- Avril 2010 & \emph{Upgrade progiciel de facturation, OS et base de donn\'ees chez SFR, Entreprise} \\
& \begin{itemize}
\item         Migration de la version Kenan BP 10.0 vers Kenan FX 12.0
\item         Migration de la cha\^{\i}ne de facturation vers le nouvel OS (Solaris 2.10) et Oracle 10g
\item         Utilisation de l'outil Comverse VUU pour extraire et convertir les donn\'ees en base Kenan FX
\end{itemize} \\

	

Juillet 2006 -- Juillet 2009 & \emph{Fusion des syst\`emes de facturation et CRM de SFR, Cegetel, 9Telecom, Club Internet, Tele2, Grand Public} \\
& \begin{itemize}
\item         Migration des donn\'ees du syst\`eme de facturation existant vers Kenan FX 11.0 et du CRM existant vers Siebel 7.0
\item         Utilisation de l'outil LCU de Comverse pour convertir et charger les donn\'ees dans une base Kenan BP
\item         Utilisation d'Informatica pour extraire et convertir les donn\'ees dans des fichiers \`a destination de Kenan BP et Siebel
\end{itemize} \\

	

Juillet 2003 -- Mai 2006 & \emph{Responsable fonctionnel de l'interface entre le CRM Siebel et le syst\`eme de billing BSCS chez SFR, Grand Public} \\
& \begin{itemize}
\item     Validation de bout en bout: impl\'ementation d'offres commerciales, valorisation d'appels, g\'en\'eration de la facture papier
\item     Am\'elioration de la qualit\'e du code avec Insure++
\item     Utilisation du CAS (API BSCS)
\item     Programmation C/C++, parser XML Xerces, utilisation de Lex \& Yacc
\end{itemize} \\

D\'ecembre 2001 -- Juin 2003 & \emph{Projet de recouvrement chez Bouygues Telecom} \\
& \begin{itemize}
\item     Programmation C/C++ sous Windows NT, Visual C++
\item     Param\'etrage du logiciel Ligis de Metamicro (www.metamicro.fr)
\end{itemize} \\

Septembre 2001 -- Novembre 2001 & \emph{Projet de collecte des appels chez Noos} \\
& \begin{itemize}
\item     D\'eveloppement bas\'e sur Object Switch (Kabira) d'une interface de collecte du syst\`eme de m\'ediation VOIP de Noos
\end{itemize} \\

Octobre 1999 -- Septembre 2001 & \emph{Migration du syst\`eme de facturation existant vers BSCS, Maroc Telecom, Grand Public} \\
& \begin{itemize}
\item     Param\'etrage initial de BSCS
\item     Programmation et maintenance de l'interface de collecte ObjectSwitch (Kabira)
\item     Programmation de diff\'erents batchs en Pro*C dans les domaines suivants : comptabilit\'e, recouvrement, roaming, offre commerciale
\item     Optimisation de requ\^etes SQL en utilisant tkprof, les outils d'explain plan d'Oracle
\end{itemize} \\

Septembre 1999 -- Octobre 1999 & \emph{Programmation de batches autour de BSCS chez Mobistar} \\
& \begin{itemize}
\item     Programmation de batches Pro*C dans le domaine de la comptabilit\'e
\end{itemize} \\

\end{longtable}
