% $Author: $ $Date: $
% $Header: $ $Name:  $
% $Revision: $
%

\documentclass[a4paper,11pt]{letter}
\usepackage{latexsym,graphicx,multicol}
\usepackage[french]{babel}
\usepackage{wrapfig}
\date{31 July 2008}
\begin{document}
Simon ELBAZ                8, rue Belgrand            92300 Levallois-Perret


Mob :    +336 14 99 02 78

E-mail : simon.elbaz@free.fr

Born $7^\mathrm{th}$ July 1975, Casablanca, married, 2 children

Engineer from Computer Sciences and Applied Mathematics at E.N.S.E.E.I.H.T (www.enseeiht.fr)

Professional Skills
\begin{itemize}
\item    Expertise fonctionnelle et technique des systèmes de Billing pour les opérateurs télécoms
\item    Validation, Recette
\item    Expertise technique ETL
\item    Analyse des besoins, conception et implémentations des solutions
\item    Utilisation avancée des bases de données Oracle
\end{itemize}

Functional Skills
\begin{itemize}
\item    Telecom Operators
\item    Billing, Call Records processing, Dunning, CRM
\item    Revenue Management
\item    Driving License
\item    Internet Protocols (IPv4, IPv6), Web 2.0
\end{itemize}

Technical Skills
\begin{itemize}
\item    Professional Software: BSCS, Kenan FX, Oracle BRM, Arbor BP, Talend, Datastage, Informatica, Zope3, Subversion, Telelogic Continuus, Siebel, Tuxedo, Valgrind, Insure++
\item    Database: Oracle, Sybase, MySQL, Postgresql, SQLite
\item    OS : Linux, Unix, NT
\item    Languages : C/C++, Python, shell, PL/SQL, Pro*C, Perl, Java, Lex, Yacc
\item    Networks : SSL, IPv6
\end{itemize}

References

CAPGEMINI, France
	

Test Team Responsible / Programmer at Capgemini, France
	

Tech Data, France
	

Marketing training (6 weeks) at Tech Data Corporation

Languages
\begin{itemize}
\item    English : fluent (TOEIC : 740/990)
\item    French : mother tongue
\end{itemize}

Experience

\begin{tabular}{|p{0.2\textwidth}|p{0.8\textwidth}|}
\hline
From / To & Description \\
\hline
January 2012 \slash Now & Data migration of driving licences to a new system, Agence Nationale des Titres Sécuris\'es \\
\hline
	

 & Test of migration process from legacy system (Service National des Permis de Conduire) to new system called Faeton 
\begin{itemize}
\item Test plan writing into HP Quality Center
\item Use of HP Quality Center to monitor test progress
\item 3 person team responsible
\end{itemize} \\
\hline
	

Octobre 2011 \slash December 2011 & Revenue Management at Air France - KLM \\
\hline
	

     & 
\begin{itemize}
\item Specification writing, programming for the Karma project within the Revenue Management department at Air France - KLM

\item Dealing with a three year archive of all Air France - KLM bookings. 2 billions of booking ticket events processed wih Perl and Talend ETL within 4 hours to extract data to feed a revenue management software
\end{itemize} \\
\hline
	

August 2011 \slash September 2011 & Data migration to upgraded inventory software, Orange \\
\hline
& \begin{itemize}
\item Test of migrated data into Torex inventory software for telco operator Orange in a 2 person team
\item ETL Talend programming
\end{itemize} \\
\hline

May 2010 \slash July 2011 & Installation of a billing system for mobile telco operator, SFR \\
	
\hline

& \begin{itemize}
\item Test of installed billing software BRM Oracle
\item Test of ETL Datastage programs
\item Call record rating C programming using BRM Oracle pipeline tool
\end{itemize} \\
	
\hline

%Août 2009 – Avril 2010
%	
%
%Upgrade progiciel de facturation, OS et base de données chez SFR, Entreprise
%	
%
%        Migration de la version Kenan BP 10.0 vers Kenan FX 12.0
%        Migration de la chaîne de facturation vers le nouvel OS (Solaris 2.10) et Oracle 10g
%        Utilisation de l'outil Comverse VUU pour extraire et convertir les données en base Kenan FX
%        Tests des données migrées et de non régression
%
%	
%
%Juillet 2006 – Juillet 2009
%	
%
%Fusion des systèmes de facturation et CRM de SFR, Cegetel, 9Telecom, Club Internet, Tele2, Grand Public
%	
%
%        Migration des données du système de facturation existant vers Kenan FX 11.0
%
%        Migration des données du CRM existant vers Siebel 7.0
%
%        Spécification des méthodes de migration
%        Utilisation de l'outil LCU de Comverse pour convertir et charger les données dans une base Kenan BP
%        Test des données migrées
%        Utilisation d'Informatica pour extraire et convertir les données dans des fichiers à destination de Kenan BP et Siebel
%
%	
%
%Juillet 2003 – Mai 2006
%	
%
%Responsable fonctionnel de l'interface entre le CRM Siebel et le système de billing BSCS chez SFR, Grand Public
%	
%
%    Responsable d'équipe (2 personnes)
%    Validation de bout en bout: implémentation d'offres commerciales, valorisation d'appels, génération de la facture papier
%    Amélioration de la qualité du code avec Insure++
%    Gestion des sources avec Continuus
%    Utilisation du CAS (API BSCS)
%    Délais tendus de livraison des évolutions
%    Programmation C/C++, parser XML Xerces, utilisation de Lex & Yacc
%
%	
%
%Décembre 2001 – Juin 2003
%	
%
%Projet de recouvrement chez Bouygues Telecom
%	
%
%    Spécification et programmation
%
%    Programmation C/C++ sous Windows NT, Visual C++
%
%    Paramétrage du logiciel Ligis de Metamicro (www.metamicro.fr)
%
%    Spécification et programmation
%
%    Programmation C/C++ sous Windows NT, Visual C++
%
%    Paramétrage du logiciel Ligis de Metamicro (www.metamicro.fr)
%
%	
%
%Septembre 2001 – Novembre 2001
%	
%
%Projet de collecte des appels chez Noos
%	
%
%    Développement basé sur Object Switch (Kabira) d'une interface de collecte du système de médiation VOIP de Noos
%    Installation et validation de l'interface de collecte VOIP chez Noos
%    Installation de softs nécessaires au fonctionnement de l'interface de collecte : Apache, PHP 
%
%	
%
%Octobre 1999 – Septembre 2001
%	Migration du système de facturation existant vers BSCS, Maroc Telecom, Grand Public
%	
%
%    Paramétrage initial de BSCS
%
%    Programmation et maintenance de l'interface de collecte ObjectSwitch (Kabira)
%
%    Programmation de différents batchs en Pro*C dans les domaines suivants : comptabilité, recouvrement, roaming, offre commerciale
%
%    Optimisation de requêtes SQL en utilisant tkprof, les outils d'explain plan d'Oracle
%
%	
%
%Septembre 1999 – Octobre 1999
%	
%
%Programmation de batches autour de BSCS chez Mobistar
%	
%
%    Programmation de batches Pro*C dans le domaine de la comptabilité
%
\end{tabular}
%
%Cursus
%
%1996-99        Elève ingénieur à l'ENSEEIHT dans le département Informatique et Mathématiques Appliquées
%
%1993-96        Classes préparatoires aux Grandes Ecoles en Mathématiques et Physique, Chimie au Lycée Carnot, Paris
%
%1993        Baccalauréat en Mathématiques et Physique avec mention Bien au Lycée Lyautey1, Casablanca, Maroc
%
%Centres d'intérêts
%
%Hobbies :
%
%Natation, lecture
%
%Levallois, le 4 juin 2012
\end{document}
