% $Author: $ $Date: $
% $Header: $ $Name:  $
% $Revision: $
%

\documentclass[a4paper,11pt]{letter}
\usepackage{latexsym,graphicx,multicol}
\usepackage[french]{babel}
\usepackage{wrapfig}
\date{31 July 2008}
\begin{document}
Simon ELBAZ                8, rue Belgrand            92300 Levallois-Perret


Mob :    +336 14 99 02 78

E-mail : simon.elbaz@free.fr

Born $7^\mathrm{th}$ July 1975, Casablanca, married, 2 children

Engineer from Computer Sciences and Applied Mathematics at E.N.S.E.E.I.H.T (www.enseeiht.fr)

Professional Skills
\begin{itemize}
\item    Testing
\item    Good knowledge of ETL
\item    Good knowledge of telco operator billing systems
\item    Customer need analysis, good programmer
\end{itemize}

Functional Skills
\begin{itemize}
\item    Telecom Operators
\item    Billing, Call Records processing, Dunning, CRM
\item    Revenue Management
\item    Driving License
\item    Internet Protocols (IPv4, IPv6), Web 2.0
\end{itemize}

Technical Skills
\begin{itemize}
\item    Professional Software: BSCS, Kenan FX, Oracle BRM, Arbor BP, Talend, Datastage, Informatica, Zope3, Subversion, Telelogic Continuus, Siebel, Tuxedo, Valgrind, Insure++
\item    Database: Oracle, Sybase, MySQL, Postgresql, SQLite
\item    OS : Linux, Unix, NT
\item    Languages : C/C++, Python, Cython, shell, PL/SQL, Pro*C, Perl, Java, Lex, Yacc
\item    Networks : SSL, IPv6
\end{itemize}

References

CAPGEMINI, France
	

Test Team Responsible / Programmer at Capgemini, France
	

Tech Data, France
	

Marketing training (6 weeks) at Tech Data Corporation

Languages
\begin{itemize}
\item    English : fluent (TOEIC : 740/990)
\item    French : mother tongue
\end{itemize}

Experience

\begin{tabular}{p{0.2\textwidth}p{0.8\textwidth}}
From / To & Description \\
January 2012 \slash Now & Data migration of driving licences to a new system, Agence Nationale des Titres Sécuris\'es \\
	

 & Test of migration process from legacy system (Service National des Permis de Conduire) to new system called Faeton 
\begin{itemize}
\item Test plan writing into HP Quality Center
\item Use of HP Quality Center to monitor test progress
\item 3 person team responsible
\end{itemize} \\
	

Octobre 2011 \slash December 2011 & Revenue Management at Air France - KLM \\

     & 
\begin{itemize}
\item Specification writing, programming for the Karma project within the Revenue Management department at Air France - KLM

\item Dealing with a three year archive of all Air France - KLM bookings. 2 billions of booking ticket events processed wih Perl and Talend ETL within 4 hours to extract data to feed a revenue management software
\end{itemize} \\
	

August 2011 \slash September 2011 & Data migration to upgraded inventory software, Orange \\
& \begin{itemize}
\item Test of migrated data into Torex inventory software for telco operator Orange in a 2 person team
\item ETL Talend programming
\end{itemize} \\

May 2010 \slash July 2011 & Install of a billing system for mobile telco operator SFR \\
	

& \begin{itemize}
\item Test of installed billing software BRM Oracle
\item Test of ETL Datastage programs
\item Call record rating C programming using BRM Oracle pipeline tool
\end{itemize} \\
	
August 2009 \slash April 2010 & Upgrade of billing software, OS and database at SFR, Entreprise\\

& \begin{itemize}
\item        Version upgrade from Kenan BP 10.0 to Kenan FX 12.0
\item        Data migration, OS upgrade (Solaris 2.10) and database upgrade (Oracle 10g)
\item        Use of Comverse tool VUU to extract and convert data between Kenan versions
\item        Tests of migrated data and processes
\end{itemize} \\


July 2006 \slash July 2009 & Billing and CRM system fusion at SFR, Cegetel, 9Telecom, Club Internet, Tele2 \\
	
& \begin{itemize}
\item         Data migration of legacy billing ssytems to Kenan FX 11.0
\item         Data migration from existing CRM system to Siebel 7.0
\item         Specification writing
\item         Use of Comverse tool LCU to convert and load data to upgraded Kenan version
\item         Test of migrated data
\item         Utilisation d'Informatica pour extraire et convertir les données dans des fichiers à destination de Kenan BP et Siebel
\end{itemize} \\

	

%Juillet 2003 – Mai 2006
%	
%
%Responsable fonctionnel de l'interface entre le CRM Siebel et le système de billing BSCS chez SFR, Grand Public
%	
%
%    Responsable d'équipe (2 personnes)
%    Validation de bout en bout: implémentation d'offres commerciales, valorisation d'appels, génération de la facture papier
%    Amélioration de la qualité du code avec Insure++
%    Gestion des sources avec Continuus
%    Utilisation du CAS (API BSCS)
%    Délais tendus de livraison des évolutions
%    Programmation C/C++, parser XML Xerces, utilisation de Lex & Yacc
%
%	
%
%Décembre 2001 – Juin 2003
%	
%
%Projet de recouvrement chez Bouygues Telecom
%	
%
%    Spécification et programmation
%
%    Programmation C/C++ sous Windows NT, Visual C++
%
%    Paramétrage du logiciel Ligis de Metamicro (www.metamicro.fr)
%
%    Spécification et programmation
%
%    Programmation C/C++ sous Windows NT, Visual C++
%
%    Paramétrage du logiciel Ligis de Metamicro (www.metamicro.fr)
%
%	
%
%Septembre 2001 – Novembre 2001
%	
%
%Projet de collecte des appels chez Noos
%	
%
%    Développement basé sur Object Switch (Kabira) d'une interface de collecte du système de médiation VOIP de Noos
%    Installation et validation de l'interface de collecte VOIP chez Noos
%    Installation de softs nécessaires au fonctionnement de l'interface de collecte : Apache, PHP 
%
%	
%
%Octobre 1999 – Septembre 2001
%	Migration du système de facturation existant vers BSCS, Maroc Telecom, Grand Public
%	
%
%    Paramétrage initial de BSCS
%
%    Programmation et maintenance de l'interface de collecte ObjectSwitch (Kabira)
%
%    Programmation de différents batchs en Pro*C dans les domaines suivants : comptabilité, recouvrement, roaming, offre commerciale
%
%    Optimisation de requêtes SQL en utilisant tkprof, les outils d'explain plan d'Oracle
%
%	
%
%Septembre 1999 – Octobre 1999
%	
%
%Programmation de batches autour de BSCS chez Mobistar
%	
%
%    Programmation de batches Pro*C dans le domaine de la comptabilité
%
\end{tabular}
%
%Cursus
%
%1996-99        Elève ingénieur à l'ENSEEIHT dans le département Informatique et Mathématiques Appliquées
%
%1993-96        Classes préparatoires aux Grandes Ecoles en Mathématiques et Physique, Chimie au Lycée Carnot, Paris
%
%1993        Baccalauréat en Mathématiques et Physique avec mention Bien au Lycée Lyautey1, Casablanca, Maroc
%
%Centres d'intérêts
%
%Hobbies :
%
%Natation, lecture
%
%Levallois, le 4 juin 2012
\end{document}
